\documentclass{tufte-handout}
\usepackage{amsmath}
\usepackage[utf8]{inputenc}
\usepackage{mathpazo}
\usepackage{booktabs}
\usepackage{microtype}

\usepackage{color}
\usepackage{listings}


\definecolor{dkgreen}{rgb}{0,0.6,0}
\definecolor{gray}{rgb}{0.5,0.5,0.5}
\definecolor{mauve}{rgb}{0.58,0,0.82}
\definecolor{gray75}{gray}{0.75}
\definecolor{light-gray}{gray}{0.5}

\lstset{
frame=,
language=java,
aboveskip=3mm,
belowskip=3mm,
showstringspaces=false,
columns=flexible,
basicstyle={\small\ttfamily},
numbers=none,
numberstyle=\tiny\color{gray},
keywordstyle=\color{blue},
commentstyle=\color{dkgreen},
stringstyle=\color{mauve},
breaklines=true,
breakatwhitespace=true
tabsize=4
}


\setcounter{secnumdepth}{3}
\renewcommand{\thesection}{}
\renewcommand{\thesubsection}{Exercise \arabic{section}.\arabic{subsection}}
\newcommand{\skipitem}{%
\stepcounter{enumi}
}

\pagestyle{empty}


\title{PRCPP week 2}
\author{Jacob Cholewa (jbec@itu.dk)}

\setcounter{section}{1}
\begin{document}
\maketitle

\section{}


\subsection{}

\begin{enumerate}
\item It takes $7,3$ seconds. The code can be seen in $SequentialFactorCount.java$
\item The code can be seen in $MyAtomicInteger.java$
\item The result is still $18,703,729$ and the runtime using $ConcurrentFactorCount.java$ is $3,5$ seconds.
\item No. The method has to be synchronized as the addAndGet operation is atomic.
\item The consumed time is the same. The \texttt{AtomicInteger} can, but does not have to be declared final

\end{enumerate}

\subsection{}

\begin{enumerate}
\item It is important as it ensured visibility of the $OneValueCache$ object between threads.
\item As the object is already immutable (The values of $lastNumber$ and $lastFactors$ can't be changed regardless of it being $final$ or not) it is important because of visibility. The $final$ keyword ensured visibility after the constructor has finished. If the $final$ keyword is not used visibility of the objects values is not ensured between threads.
\end{enumerate}

\subsection{}
\textit{Please see the code implemented in $ConcurrentFactorHistogram.java$}
\begin{enumerate}
\item $count$ should be $final$ as it will never change and because it ensures visibility. $increment$ and $getCount$ needs to be $synchronized$ to ensure visibility and because the increment operation is atomic. $getSpan$ does not have to be synchronized as $counts$ is final.
\skipitem
\item There is a slight advantage of using $AtomicIntegers$ with a time difference of between $100ms$ and $200ms$. 
\item The version using $Histogram4$ return a correct result. ($diff$ against $check.out$)
\item For $Histogram2$ I return a clone of the internal array. For $Histogram3$ and $Histogram4$ i copy the values into a new int array and return that. All three provides a snapshot.
\end{enumerate}

\subsection{}
\begin{enumerate}
\item Please see the code implemented in $TestCache.java$
\item When $Memoizer1$ is used $Factorizer$ is called $115000$ times and it takes real $~18,0s$, user $~18,3s$ and sys $~0,7s$.
\item When $Memoizer2$ is used $Factorizer$ is called $156047$ times and it takes real $~13,1s$, user $~41,8s$ and sys $~0,3s$.
\item When $Memoizer3$ is used $Factorizer$ is called $116280$ times and it takes real $~11,2s$, user $~25,2s$ and sys $~1,1s$.
\item When $Memoizer4$ is used $Factorizer$ is called $115000$ times and it takes real $~11,2s$, user $~24,9s$ and sys $~1,0s$.
\item When $Memoizer5$ is used $Factorizer$ is called $115000$ times and it takes real $~11,6s$, user $~25,0s$ and sys $~1,0s$.
\end{enumerate}


\end{document}