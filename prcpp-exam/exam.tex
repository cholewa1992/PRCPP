\documentclass{ituhandin}

\coursename{Practical Concurrent and Parallel Programming}
\fullname{Jacob B. Cholewa}
\when{January 2016}
\initials{jbec}
\coursecode{PRCPP}

\begin{document}
\maketitlepage
\signpage

\chapter{} %1

\section{}
The output for \texttt{TestLocking0.java} clearly indicates that it is not thread-safe.
\begin{lstlisting}[language={}, frame={}]
mac610262:src jbec$ java TestLocking0
Sum is 1505632.000000 and should be 2000000.000000
mac610262:src jbec$ java TestLocking0
Sum is 1490208.000000 and should be 2000000.000000
mac610262:src jbec$ java TestLocking0
Sum is 1497894.000000 and should be 2000000.000000
mac610262:src jbec$ java TestLocking0
Sum is 1505498.000000 and should be 2000000.000000
\end{lstlisting}

The actual sum deviates from the expected sum leading me to believe that a race condition occurs in the code.

\section{}
The problem is that while the \texttt{addInstance} method locks the object instance, \texttt{addStatic} locks the class. Hence the field \texttt{sum} is not guarded by the same lock. This allows for multiple threads to simultaneously access the \texttt{sum} variable, therefore not upholding mutual exclusion, causing the race condition.

\section{}
A simple solution is to change \texttt{addInstance} so that it is guarded by the class lock used by the static synchronized methods
\begin{lstlisting}[frame={}]
public void addInstance(double x) {
    synchronized(Mystery.class){
        sum += x;
    }
}
\end{lstlisting}
This now ensures mutual exclusion as \texttt{sum} is now guarded by the same lock. Rerunning the code shows that the expected result and the actual result are now the same.

\begin{lstlisting}[language={}, frame={}]
mac610262:src jbec$ java TestLocking0
Sum is 2000000.000000 and should be 2000000.000000
mac610262:src jbec$ java TestLocking0
Sum is 2000000.000000 and should be 2000000.000000
mac610262:src jbec$ java TestLocking0
Sum is 2000000.000000 and should be 2000000.000000
mac610262:src jbec$ java TestLocking0
Sum is 2000000.000000 and should be 2000000.000000
\end{lstlisting}

\chapter{} %2
\section{}
The simplest way would be to make a synchronized version that guards \texttt{items} and \texttt{size} with an instance lock. This would ensure safe concurrent access to the arraylist.
\section{}\label{sec:lock}
While the näive approach described above makes the arraylist threadsafe, it does not allow parallel access and thus doesn't scale. Actually I expect the synchronized version to perform significantly worse when used by many threads compared to only a single thread.
\section{}
A simple answer to why the purposed pattern is not threadsafe is found in the sample given in the assignment. The example clearly shows that both \texttt{add} and \texttt{set} accesses \texttt{items} and \texttt{size}. As the methods uses different locks, concurrent access to \texttt{items} and \texttt{size} can occur making it not threadsafe.

\moreFancyQuote{When thread $A$ executes a synchronized block, and subsequently thread $B$ enters a synchronized block guarded by the same lock, the values of variables that were visible to $A$ prior to releasing the lock are guaranteed to be visible to $B$ upon acquiring the lock.}{\citet[p. 37]{goetz2006java}}

However, because the methods uses different locks, visibility is not guaranteed.
\section{}
While it is possible that a version might exist that makes this threadsafe, it will still require mutual exclusion when at least writing to \texttt{items} and \texttt{size}. I don't see many other (simple) ways than to fully lock both \texttt{items} and \texttt{size} when accessed. Thus it won't make much sense to have a lock for the methods if we either way have to lock the only two shared fields. Then we could as well just only lock those instead.

\chapter{} %3
\section{}
The \texttt{totalSize} field can be made threadsafe by either using an \texttt{AtomicInteger} or by having \texttt{totalSize} guarded by a static lock object. The following snippet shows how it would be implemented in the code using an \texttt{AtomicInteger}

\begin{lstlisting}[frame={}]
private static AtomicInteger totalSize = new AtomicInteger();

public boolean add(double x) {
    if (size == items.length) {
        ...
    }
    items[size] = x;
    size++;
    totalSize.incrementAndGet();
    return true;
}

public static int totalSize() {
    return totalSize.get();
}
\end{lstlisting}

\section{}
The \texttt{allLists} field can be make threadsafe by guarding it with a static lock object. It can be implemented in code in following way.
\begin{lstlisting}[frame={}]
private static HashSet<DoubleArrayList> allLists = new HashSet<>();
private static Object ListsLock = new Object();

public DoubleArrayList() {
    synchronized(ListsLock){
        allLists.add(this);
    }
}

public static HashSet<DoubleArrayList> allLists() {
    synchronized(ListsLock){
        return allLists;
    }
}
\end{lstlisting}

\chapter{} %4
\chapter{} %5
\section{} %wrapper
\section{} %with 40

\begin{lstlisting}[language={},frame={}]
mac610262:src jbec$ java SortingPipeline
0.1 1.1 2.1 3.1 4.1 5.1 6.1 7.1 8.1 9.1 10.1 11.1 12.1 13.1 14.1 15.1 16.1 17.1 18.1 19.1 20.1 21.1 22.1 23.1 24.1 25.1 26.1 27.1 28.1 29.1 30.1 31.1 32.1 33.1 34.1 35.1 36.1 37.1 38.1 39.1
\end{lstlisting}
\section{} %with 100000
\begin{lstlisting}[language={},frame={}]
mac610262:src jbec$ java SortingPipeline
# OS:   Mac OS X; 10.11; x86_64
# JVM:  Oracle Corporation; 1.8.0_60
# CPU:  null; 8 "cores"
# Date: 2016-01-11T14:45:02+0100
Sorting pipe                        125.9 ms       1.71          4
\end{lstlisting}
\chapter{} %6
\chapter{} %7
\chapter{} %8
\chapter{} %9
\chapter{} %10
\chapter{} %11
\chapter{} %12

\chapter*{Example}

This is code in a box

\begin{lstlisting}[caption=This is a caption]
\end{lstlisting}


This is code in the free

\begin{lstlisting}[frame={}]
\end{lstlisting}

\bibliographystyle{plainnat}
\bibliography{bibliography}

\label{LastPage}
\end{document}
